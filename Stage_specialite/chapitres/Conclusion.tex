%% Conclusion [1-2 pages]
%% Le sujet a-t-il été entièrement traité ? Que reste-t-il à faire ? Quelles sont les évolutions possibles ?, ...
%TODO% changer
% Difficule de faire la gestion (pour la suite,...) donc wordpress
% wordpress ne peut pas etre modofier 
% l'environnement utilise Microsoft 
% Manuale est souvent utilisé
% incovient du wordpress: ne peut pas changer l'affichage, la mise en page etc...

La conclusion sera scindée en trois parties. Dans un premier temps, je présenterai ce qui a été réalisé vis à vis du cahier des charges. Dans un second temps, je donnerai ce qu'il reste à faire et les évolutions possibles et enfin je tirerai un bilan personnel vis à vis de ce stage.

\section{Les réalisations}

Nous avions un cahier des charges détaillé des fonctionnalités à implanter (cf \ref{CDC}). 
Le cahier des charges était divisé en deux parties et ce stage nous a permis de réaliser entièrement la première partie, qui concernait l’espace intranet du site de la société (le \texttt{wordpress}).\\

L'application est entièrement fonctionnelle et est maintenant en test pour être mise en ligne en Octobre 2012. Les performances, malgré le modèle, sont beaucoup plus puissantes par rapport aux précedents, en fonction de la mise en page du formulaire, la facilité pour l’accès de la base de données, etc.\\

Nous avons également proposé les solutions pour la deuxième partie, notamment la gestion de l'interface de cette application web qui devra intégrée dans le site web du Windeo Green Futur.\\

Par ailleurs, nous avons maintenu la documentation et l'avons adaptée aux changements que nous avons apportés. Cependant, nous avons manqué de temps pour la rendre plus complète.\\

\section{Les améliorations à effectuer}

La plupart des améliorations à effectuer sont situées dans la deuxième partie du Cahier de charge. Nous n'avons ainsi pas eu le temps de générer le menu de l'application via la base de données ou encore migrer les fonctionnalités suiffantes. Certains affichages de l'interface restent également à améliorer.

\section{Bilan personnel}

A l'issue de ces 10 semaines de stage je peux dire que cette expérience a été très bénéfique pour moi. J'ai pu mettre en oeuvre mes compétences acquises lors de ma formation dans le département ASI et en acquérir de nouvelles.\\

Le fait d'avoir intégré une équipe de travail, d'avoir participé à des reunions d'avancement ou de réflexions mais aussi d'avoir d\^u respecter des contraintes temporelles pour certaines livraisons ont été à mes yeux très enrichissant pour mon expérience professionnelle.\\

Sur le plan technique, ce stage m'a permis de me rendre compte de la difficulté de reprise d'une application web lorsque peu de documentation est disponible. Construire une site web complète s'avère être néanmoins une très bonne expérience pour comprendre les mécanismes d'une application malgré son coût énorme en temps. Je me suis également rendu compte que les Technologies Web sont certes intéressantes mais assez répétitives. \\

Sur le plan personnel, bien que je travaille depuis quelques années assez fréquemment avec PHP, ce stage m'a permis encore une fois d'expérimenter la dynamique et le travail en équipe. Ceci est un réel plus et je ne pense pas que ce stage aurait été aussi bien réussi sans cela. 
